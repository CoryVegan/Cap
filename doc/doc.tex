\documentclass[10pt, oneside]{article}   	% use "amsart" instead of "article" for AMSLaTeX format
\usepackage{geometry}                		% See geometry.pdf to learn the layout options. There are lots.
\geometry{letterpaper}                   		% ... or a4paper or a5paper or ... 
%\geometry{landscape}                		% Activate for for rotated page geometry
%\usepackage[parfill]{parskip}    		% Activate to begin paragraphs with an empty line rather than an indent
\usepackage{graphicx}				% Use pdf, png, jpg, or eps§ with pdflatex; use eps in DVI mode
								% TeX will automatically convert eps --> pdf in pdflatex		
\usepackage{amssymb}
\usepackage{amsmath}



\begin{document}


\footnotesize
\begin{tabular}{llll}
Path & Parameter & Value & Units \\
\texttt{material\_properties.separator\_density}               & $\rho^{separator}$    & 1.2528e3  & [kg/m$^3$] \\
\texttt{material\_properties.electrode\_density}               & $\rho^{electrode}$    & 0.93e3    & [kg/m$^3$] \\
\texttt{material\_properties.collector\_density}               & $\rho^{collector}$    & 2.7e3     & [kg/m$^3$] \\
\texttt{material\_properties.separator\_heat\_capacity}        & $C_p^{separator}$     & 3.1404e3  & [J/K] \\
\texttt{material\_properties.electrode\_heat\_capacity}        & $C_p^{electrode}$     & 1.34e3    & [J/K] \\
\texttt{material\_properties.collector\_heat\_capacity}        & $C_p^{collector}$     & 0.89815e3 & [J/K] \\
\texttt{material\_properties.separator\_thermal\_conductivity} & $\lambda^{separator}$ & 0.0019e2  & [W/m$\cdot$K] \\
\texttt{material\_properties.electrode\_thermal\_conductivity} & $\lambda^{electrode}$ & 0.0011e2  & [W/m$\cdot$K] \\
\texttt{material\_properties.collector\_thermal\_conductivity} & $\lambda^{collector}$ & 2.37e2    & [W/m$\cdot$K] \\
\texttt{operating\_conditions.charge\_potential}           & $U_{charge}$    & 2.2     & [V] \\
\texttt{operating\_conditions.discharge\_potential}        & $U_{discharge}$ & 1.1     & [V] \\
\texttt{operating\_conditions.charge\_current\_density}    & $I_{charge}$    & 324.65  & [A/m${2}$] \\
\texttt{operating\_conditions.discharge\_current\_density} & $I_{discharge}$ & -324.65 & [A/m${2}$] \\
\texttt{material\_properties.specific\_capacitance}             & $aC$                   & 86.0e6    & [F/m$^3$] \\
\texttt{material\_properties.separator\_void\_volume\_fraction} & $\epsilon^{separator}$ & 0.6       & [1] \\
\texttt{material\_properties.electrode\_void\_volume\_fraction} & $\epsilon^{electrode}$ & 0.67      & [1] \\
\texttt{material\_properties.electrolyte\_conductivity}         & $\sigma_l$             & 0.067     & [S/m] \\
\texttt{material\_properties.solid\_phase\_conductivity}        & $\sigma_s$             & 52.1      & [S/m] \\
\texttt{geometry.electrode\_width} & $w^{electrode}$ & 50.0e-6 & [m] \\
\texttt{geometry.separator\_width} & $w^{separator}$ & 25.0e-6 & [m] \\
\texttt{geometry.collector\_width} & $w^{collector}$ &  5.0e-6 & [m] \\
\texttt{geometry.sandwich\_height} & $h^{sandwich}$  & 25.0e-6 & [m] \\
\texttt{operating\_conditions.max\_cycles}      &  & 100  & [1] \\
\texttt{operating\_conditions.relaxation\_time} & $t_{pause}$     &  5.0 & [s] \\
\\
\texttt{upper\_heat\_transfer\_coefficient} & $h$   & 8.0e-2 & [W/m$^2\cdot$K] \\
\texttt{lower\_heat\_transfer\_coefficient} & $h$   & 0.0    & [W/m$^2\cdot$K] \\
\texttt{left\_heat\_transfer\_coefficient}  & $h$   & 0.0    & [W/m$^2\cdot$K] \\
\texttt{right\_heat\_transfer\_coefficient} & $h$   & 0.0    & [W/m$^2\cdot$K] \\
\texttt{upper\_ambient\_temperature}        & $T_a$ & 0.0    & [K] \\
\texttt{lower\_ambient\_temperature}        & $T_a$ & 0.0    & [K] \\
\texttt{left\_ambient\_temperature}         & $T_a$ & 0.0    & [K] \\
\texttt{right\_ambient\_temperature}        & $T_a$ & 0.0    & [K] \\
\texttt{.initial\_time}    & $t_0$           &  0.0 & [s] \\
\texttt{.final\_time}      & $t_1$           & 30.0 & [s] \\
\texttt{.time\_step}       & $\Delta t$      &  0.1 & [s] \\
\texttt{.mesh\_resolution} &  & [1 -- 10] \\
\texttt{.dim}              &  & 2 \\
\end{tabular}


\begin{tabular}{llll}
\texttt{output.max\_temperatures} & $\max{T(t_i)}$ & [K] \\
\texttt{output.Joule\_heating} & $Q(t_i)$ & [W] \\
\texttt{output.voltage} & $U$ & [V] \\
\texttt{output.current} & $I$ & [A] \\
\texttt{output.time} & $t_i$ & [s] \\
\texttt{output.cycle} &  & [1] \\
\texttt{output.power\_density} & $P$ & [W/m$^3$] \\
\texttt{output.energy\_density} & $E$ & [J/m$^3$] \\
\end{tabular}

\section{Geometry}

\section{Governing equations}
%----------------------------------------------------------------------------- 
% ELECTROCHEMICAL
%----------------------------------------------------------------------------- 
Governing equations for the electrolyte potential $\Phi_l$ and the solid phase
potential $\Phi_s$ are
\begin{align}
aC (\dot{\Phi}_l - \dot{\Phi}_s) &= \nabla \cdot (\sigma_l \nabla \Phi_l) \text{ on } \Omega \\
aC (\dot{\Phi}_s - \dot{\Phi}_l) &= \nabla \cdot (\sigma_s \nabla \Phi_s) \text{ on } \Omega
\end{align}
$aC$ is the specific capacitance, 
$\sigma_l=\sigma_l\varepsilon^{1.5}$ and $\sigma_s=\sigma_s(1-\varepsilon)^{1.5}$ 
are the electrolyte and solid phase electrical conductivities, 
$\varepsilon$ is the porous medium void volume fraction.

anode 

cathode

Boundary conditions, choose from either potentiostatic charge/discharge
(Dirichlet)
\begin{equation}
\Phi_s = u_{charge/discharge} \text{ on } \Gamma
\end{equation}
or galvanostatic charge/discharge (Neumann)
\begin{equation}
- \sigma_s \frac{\partial \Phi_s}{\partial n} = \pm i_{charge/discharge} \text{ on } \Gamma
\end{equation}

%----------------------------------------------------------------------------- 
% THERMAL
%----------------------------------------------------------------------------- 
The temperature field $\Theta$ obeys
\begin{equation}
\rho C_p \dot{\Theta} = \nabla \cdot (\lambda \nabla \Theta) + q \text{ on } \Omega
\end{equation}
Material properties $\rho$, $C_p$ and $\lambda$ are density, heat capacity and
thermal conductivity.
The last two terms account for Joule heating.
Robin boundary condition with heat transfer coefficient $h$
\begin{equation}
- \lambda \frac{\partial \Theta}{\partial n} = h (\Theta - \Theta_{ambient}) \text{ on } \Gamma
\end{equation}

\end{document}  
